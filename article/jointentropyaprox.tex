\documentclass[journal]{IEEEtran}

%\usepackage[brazil]{babel}
%\usepackage[T1]{fontenc}

\usepackage{theorem}        %%Lo agregue yo <========================================
\usepackage{algorithmic}        %%Lo agregue yo <========================================

\setcounter{secnumdepth}{7}

\newtheorem{theorem}{Theorem}%[section]
%\newtheorem{acknowledgement}[theorem]{Acknowledgement}
%\newtheorem{algorithm}[theorem]{Algorithm}
%\newtheorem{axiom}[theorem]{Axiom}
%\newtheorem{case}[theorem]{Case}
%\newtheorem{claim}[theorem]{Claim}
%\newtheorem{conclusion}[theorem]{Conclusion}
%\newtheorem{condition}[theorem]{Condition}
%\newtheorem{conjecture}[theorem]{Conjecture}
%\newtheorem{criterion}[theorem]{Criterion}
%\newtheorem{exercise}[theorem]{Exercise}
%\newtheorem{notation}[theorem]{Notation}
%\newtheorem{problem}[theorem]{Problem}
%\newtheorem{proposition}[theorem]{Proposition}
%\newtheorem{remark}[theorem]{Remark}
%\newtheorem{solution}[theorem]{Solution}
%\newtheorem{summary}[theorem]{Summary}

\newtheorem{definition}[theorem]{Definition}
\newtheorem{example}[theorem]{Example}
\newtheorem{lemma}[theorem]{Lemma}
\newenvironment{proof}[1][Proof]{\textbf{#1.} }{\ \rule{0.5em}{0.5em}}
\newtheorem{corollary}[theorem]{Corollary}
\newenvironment{algorithm}[1][Algorithm]{\textbf{#1.} }{}

\usepackage{amssymb}
\usepackage{graphicx}
\usepackage{amsmath}
\usepackage{psfrag}

\usepackage{accents}
%\usepackage[none]{hyphenat}

\usepackage[usenames,dvipsnames,svgnames,table]{xcolor}

\hyphenation{bet-ween re-pre-sen-ting} %
\sloppy

\begin{document}

\title{Approximate Joint Entropy for $M$ Correlated Binary Sources }


\author{Fernando P. Rivera 
\thanks{Manuscript received XXXX XX, 2014; revised XXXXX XX, 2014.}
\thanks{---------- ---------- ---------- ---------- ---------- ---------- ---------- ---------- ---------- ---------- ---------- }%%%%Fernando P. Rivera is with Department of Communications, State University of Campinas, Campinas, SP, Brazil. Email:fpujaico@decom.fee.unicamp.br. }
\thanks{---------- ---------- ---------- ---------- ---------- ---------- ---------- ---------- ---------- ---------- ---------- }}%%%%Jaime Portugheis   is with Faculty of Technology       , State University of Campinas, Limeira , SP, Brazil. Email:jaime@ft.unicamp.br  .}}

\markboth{IEEE Communications Letters,~Vol.~X,
No.~XX,~XXXXX~XXX}{Shell \MakeLowercase{\textit{et al.}}: Bare
Demo of IEEEtran.cls for Journals}

% make the title area
\maketitle
%%%%%%%\IEEEpeerreviewmaketitle


\begin{abstract}
This paper proposes a method for to get an approximation of the joint entropy
of $M$ correlated binary sources; the approximation gets a linear calculation complexity 
in relation to the number of sources. 
Here, it is used a correlating model with a common
binary source that through $M$ binary symmetric channels in parallel
to obtain  the $M$ correlated binary sources. 

\end{abstract}

\begin{keywords}
Multiple correlated sources, large scale sensor networks, joint entropy.
\end{keywords}

\IEEEpeerreviewmaketitle
%%%%%%%%%%%%%%%%%%%%%%%%%%%%%%%%%%%%%%%%%%%%%%%%%%%%%%%%%%%%%%%%%%%%%%%%%%%%%%%%%%%%%%%
%%%%%%%%%%%%%%%%%%%%%%%%%%%%%%%%%%%%%%%%%%%%%%%%%%%%%%%%%%%%%%%%%%%%%%%%%%%%%%%%%%%%%%%
%%%%%%%%%%%%%%%%%%%%%%%%%%%%%%%%%%%%%%%%%%%%%%%%%%%%%%%%%%%%%%%%%%%%%%%%%%%%%%%%%%%%%%%
\section{Introduction}
\label{sec:Intro}

 In the work seen in \cite{fernando}, they are presented equations for the calculus of two cases the 
 optimal minimum rates in joint source-channel coding of a set of correlated sources;
 thus, were showed methods for to get the minimal,  common rate and  sum rate.
 These  rates fulfill the Slepian-Wolf 
 \cite{slepian} and channel capacity limit \cite{cover} theorems.
 %This sources transmit your codified information across $M$ orthogonal
 %channels to a joint decoder. 
 The results show that, the calculus of 
 the optimal rates of $M$ correlated sources grows in complexity exponentially 
 with $M$.  
 
 In this paper, is assumed the same system 
 model and analyzed the same two cases that 
 in \cite{fernando}, with the additional restriction that the 
 sources are so far of the joint decoder, in comparing with the distance 
 between sensors, so that the channel capacities in all channels are 
 approximately same. On the other side, the correlation between sources is
 assumed as random or with spatial correlation \cite{corrspatial},
 in contrast with studied in \cite{ceobinary1,ceobinary2}, where the correlation
 between any source pair is same. Other restriction, it 
 is the use of a specific model of correlated sources; being used a model similar to seen
 in \cite{ceobinary1,ceobinary2}; where $M$
 correlated sources are generates passing a common source across $M$ binary 
 symmetric channels ($BSC$).
 Using these considerations, it can be deduced: for an optimal common rate a method 
 with a calculus complexity that grows linearly with the number of sources, 
 and for optimal sum rate a calculus method of \textcolor{red}{very low complexity
 when compared of common rate case}. 
 
This paper is organized as follows. The model system and some definitions 
used in this work are presented in Section \ref{sec:SystemModel}, a brief review of the
work in \cite{fernando} with a new solution method for to get the optimal 
common rate is presented in Section \ref{sec:Optimo}, in this line a new method for to get
the optimal sum rate is described in Section \ref{sec:sumrate}. Some demonstrations
need for to solve the last sections are presented in
Section \ref{sec:Appendix} and Section \ref{sec:Conclusions} concludes the paper 
with some final remarks.

%%%%%%%%%%%%%%%%%%%%%%%%%%%%%%%%%%%%%%%%%%%%%%%%%%%%%%%%%%%%%%%%%%%%%%%%%%%%%%
%%%%%%%%%%%%%%%%%%%%%%%%%%%%%%%%%%%%%%%%%%%%%%%%%%%%%%%%%%%%%%%%%%%%%%%%%%%%%%
%%%%%%%%%%%%%%%%%%%%%%%%%%%%%%%%%%%%%%%%%%%%%%%%%%%%%%%%%%%%%%%%%%%%%%%%%%%%%%
\section{System Model and Definitions} 
\label{sec:SystemModel}



%%%%%%%%%%%%%%%%%%%%%%%%%%%%%%%%%%%%%%%%%%%%%%%%%%%%%%%%%%%%%%%%%%%%%%%%%%%%%%
\section{Final Remarks and Conclusions} 
\label{sec:Conclusions}
In this letter, we considered joint source-channel coding of correlated
sources transmitted over orthogonal

%%%%%%%%%%%%%%%%%%%%%%%%%%%%%%%%%%%%%%%%%%%%%%%%%%%%%%%%%%%%%%%%%%%%%%%%%%%%%%
%%%%%%%%%%%%%%%%%%%%%%%%%%%%%%%%%%%%%%%%%%%%%%%%%%%%%%%%%%%%%%%%%%%%%%%%%%%%%%
%%%%%%%%%%%%%%%%%%%%%%%%%%%%%%%%%%%%%%%%%%%%%%%%%%%%%%%%%%%%%%%%%%%%%%%%%%%%%%
\section*{Acknowledgment}


%%%%%%%%%%%%%%%%%%%%%%%%%%%%%%%%%%%%%%%%%%%%%%%%%%%%%%%%%%%%%%%%%%%%%%%%%%%%%%
%\appendix
\section{Appendix} \label{sec:Appendix}

%%%%%%%%%%%%%%%%%%%%%%%%%%%%%%%%%%%%%%%%%%%%%%%%%%%%%%%%%%%%%%%%%%%%%%%%%%%%%%
%%%%%%%%%%%%%%%%%%%%%%%%%%%%%%%%%%%%%%%%%%%%%%%%%%%%%%%%%%%%%%%%%%%%%%%%%%%%%%
%%%%%%%%%%%%%%%%%%%%%%%%%%%%%%%%%%%%%%%%%%%%%%%%%%%%%%%%%%%%%%%%%%%%%%%%%%%%%%
\begin{thebibliography}{99}

\bibitem{fernando} Pujaico, F.; Portugheis, J.,
``Optimal Rate for Joint Source-Channel Coding of Correlated Sources Over 
Orthogonal Channels,'' Communications Letters, 2014.

\bibitem{ceobinary1} Haghighat, J.; Behroozi, Hamid; Plant, D.V., 
``Iterative joint decoding for sensor networks with binary CEO model,'' 
Signal Processing Advances in Wireless Communications, 2008. SPAWC 2008. 
IEEE 9th Workshop on , vol., no., pp.41,45, 6-9 July 2008.

\bibitem{ceobinary2} Ferrari, G.; Martalo, M.; Abrardo, A.; Raheli, R., 
``Orthogonal multiple access and information fusion: How many observations are needed?,'' 
Information Theory and Applications Workshop (ITA), 2012 , vol., no., pp.311,320, 5-10 Feb. 2012.


\end{thebibliography}

\end{document}


%%%%%%%%%%%%%%%%%%%%%%%%%%%%%%%%%%%%%%%%%%%%%%%%%%%%%%%%%%%%%%%%%%%%%%%%%%%%%%
%%%%%%%%%%%%%%%%%%%%%%%%%%%%%%%%%%%%%%%%%%%%%%%%%%%%%%%%%%%%%%%%%%%%%%%%%%%%%%
%%%%%%%%%%%%%%%%%%%%%%%%%%%%%%%%%%%%%%%%%%%%%%%%%%%%%%%%%%%%%%%%%%%%%%%%%%%%%%
%%%%%%%%%%%%%%%%%%%%%%%%%%%%%%%%%%%%%%%%%%%%%%%%%%%%%%%%%%%%%%%%%%%%%%%%%%%%%%
%%%%%%%%%%%%%%%%%%%%%%%%%%%%%%%%%%%%%%%%%%%%%%%%%%%%%%%%%%%%%%%%%%%%%%%%%%%%%%
%%%%%%%%%%%%%%%%%%%%%%%%%%%%%%%%%%%%%%%%%%%%%%%%%%%%%%%%%%%%%%%%%%%%%%%%%%%%%%
%%%%%%%%%%%%%%%%%%%%%%%%%%%%%%%%%%%%%%%%%%%%%%%%%%%%%%%%%%%%%%%%%%%%%%%%%%%%%%
%%%%%%%%%%%%%%%%%%%%%%%%%%%%%%%%%%%%%%%%%%%%%%%%%%%%%%%%%%%%%%%%%%%%%%%%%%%%%%
%%%%%%%%%%%%%%%%%%%%%%%%%%%%%%%%%%%%%%%%%%%%%%%%%%%%%%%%%%%%%%%%%%%%%%%%%%%%%%
%%%%%%%%%%%%%%%%%%%%%%%%%%%%%%%%%%%%%%%%%%%%%%%%%%%%%%%%%%%%%%%%%%%%%%%%%%%%%%
%%%%%%%%%%%%%%%%%%%%%%%%%%%%%%%%%%%%%%%%%%%%%%%%%%%%%%%%%%%%%%%%%%%%%%%%%%%%%%
%%%%%%%%%%%%%%%%%%%%%%%%%%%%%%%%%%%%%%%%%%%%%%%%%%%%%%%%%%%%%%%%%%%%%%%%%%%%%%

\grid
